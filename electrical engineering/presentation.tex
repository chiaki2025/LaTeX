\documentclass[11pt,aspectratio=169]{beamer}
\usetheme{Copenhagen}
\usecolortheme{beaver}
\usepackage[utf8]{inputenc}
\usepackage{graphicx}
\usepackage{amsmath}
\usepackage{gensymb}
\usepackage{setspace}
\onehalfspacing
\title{AC Machines Problems and Solutions}
\author{Group 14}
\date{May 2024}
\begin{document}
\maketitle
\begin{frame}{Problem 16.4}
A $60$ $Hz$ induction motor is needed to drive 
a load at approximately $850$ $rpm$. How many 
poles should the motor have? What is the 
slip of this motor for a speed of $850$ $rpm$?
\\ \textbf{Solution:}
    $$P=\frac{120f}{n_{s}}=\frac{120.60}{850}\approx8,47$$
    $\Rightarrow$ The motor should have $10$ poles, the slip:
    $$s=\frac{\omega_{s}-\omega_{m}}{\omega_{s}}=\frac{\frac{120.60}{10}-850}{\frac{120.60}{10}}=\frac{5}{90}$$
\end{frame}
\begin{frame}{Problem 16.6}
    Prepare a table that shows synchronous 
    speeds for three-phase induction motors 
    operating at $50$ $Hz$. Consider motors having 
    eight or fewer poles. Repeat for $400$ $Hz$ 
    motors.
    \\ \textbf{Solution:}
    \\With $f=50 Hz$, we have:
    \begin{center}
       \begin{tabular}{ c | c}
        $P$ & $n_{s}$  \\ 
        2 & 3000  \\  
        4 & 1500\\
        6 & 1000\\
        8 & 750\\
       \end{tabular}
       \end{center}
    \end{frame}
        
    \begin{frame}{Problem 16.6}
    With $f=400 Hz$, we have:
       \begin{center}
          \begin{tabular}{ c | c}
           $P$ & $n_{s}$  \\ 
           2 & 24000  \\  
           4 & 12000\\
           6 & 8000\\
           8 & 6000\\
          \end{tabular}
          \end{center}
\end{frame}
\begin{frame}{Problem 16.10}
    A $10$ $hp$ six-pole $60$ $Hz$ three-phase induction motor runs at 1160 rpm under full-load 
      conditions. Determine the slip and the frequency of the rotor currents at full load.  Also estimate the speed if the load torque 
      drops in half.
      \\ \textbf{Solution:}\\
      Synchronous speed:
      $$n_{s}=\frac{120f}{p}=\frac{120.60}{6}=1200$$
      Slip:
      $$s=\frac{n_{s}-n_{m}}{n_{s}}.100\%=3,33\%$$
\end{frame}
    
\begin{frame}{Problem 16.10}
      Frequency of rotor current:
      $$f_{rot}=3f=\frac{3,33\%}{100\%}.60=2(Hz)$$
      We have:
      $$T_{1}=\frac{T_{load}}{2}\Rightarrow s_{1}=\frac{s}{2}=1,66\%$$
      New speed:
      $$s_{1}=\frac{n_{s}-n_{m}}{n_{s}}.100\%\Rightarrow n_{m1}=n_{s}-\frac{s_{1}n_{s}}{100\%}=1200-\frac{1,665\%.1200}{100\%}=1180 (rpm)$$ 
    
\end{frame}
\begin{frame}{Problem 16.14}
    A two-pole $60$ $Hz$ induction motor produces an output power
of $3$ $hp$ at a speed of $1700$ $rpm$. With no load, the speed is $1798$ $rpm$. Assume that the
rotational torque loss is independent of speed. Find the rotational power loss at $1700$ $rpm$.
\end{frame}
\begin{frame}{Problem 16.14}
   \textbf{Solution:} From the problem discription we have:
$P=2, f=60Hz, P_{out}=5Hp, n_{m1}=3500rpm, n_{n0-load}=3598 rpm, T_{loss}= constant$.
\\The output power in Watt is:
$$P_{out}=5.746=3730$$
\\The output power:
$$P_{out}=T_{out}.\omega.m_{1}=T_{out}.n.m_{1}.\frac{2\pi}{60}$$
\\The output torque is:
$$T_{out}=\frac{P_{out}}{nm_{1}}.\frac{60}{2\pi}=\frac{3730}{3500}.\frac{60}{2\pi}=10,177$$
\end{frame}
\begin{frame}{Problem 16.14}
    Furthermore, we know that $T_{out}$ is equal to:
$$T_{out}=T_{dev}-T_{rot}\Rightarrow T_{dev}=T_{out}+T_{rot}$$
From the previous we know that the developed torque is proportional to slip for
the small values of slip. In our case the slip is:
\begin{equation*}
s=\frac{n_{s}-n_{m1}}{n_{s}}.100\% \Rightarrow n_{s}=\frac{120f}{p}=3600 rpm
\end{equation*}
\begin{equation*}
s=\frac{3600-3500}{3600}.100\%=2,78\%=0,0278
\end{equation*}
\end{frame}
\begin{frame}{Problem 16.14}
    We can conclude that the previous claim is correct. Then we can write:
\begin{align*}
T_{dev}&=Ks\\
T_{out}+T_{rot}&=0,0278K\\
10,177+T_{rot}&=0,0278K
\end{align*}
Under no load condition, we have:
$$n_{s}=3600 rpm, n_{n0-load}=3598 rpm$$
$$T_{out}=T_{dev}-T_{rot}=0\Rightarrow T_{dev}=T_{rot}$$
\end{frame}
\begin{frame}{Problem 16.14}
    The developed torque is proportional to slip, so we can write again:
\begin{align*}
T_{dev}=T_{rot}=Ks=K\frac{n_{s}-n_{n0}-load}{n_{s}}=K.\frac{3600-3598}{3600}=\frac{2K}{3600}
\end{align*}
Substituting $T_{dev}$ value, we have:
$$10.177+\frac{2K}{3600}=0,0278K\Rightarrow K=373,55$$
\end{frame}
\begin{frame}{Problem 16.14}
Then, the torque loss is:
$$T_{rot}=\frac{2K}{3600}=0,2075$$
Finally, the rotational power is:
$$P_{rot}=T_{rot}.\omega m_{1}=T_{rot}.nm_{1}\frac{2\pi}{60}=0,2075.3500.\frac{2\pi}{60}=76,05W$$
\end{frame}
\begin{frame}{Problem 16.15}
    A certain four-pole $230-V-rms$ $60$ $Hz$ delta connected three-phase induction motor has
    $R_{s} = 1\Omega, X_{s} = 1,5\Omega, X_{m}=40\Omega,R_{r}^{'}=0,5\Omega,X_{r}^{'}=0,8\Omega$.    
    Under load, the machine operates at $1740$ 
    $rpm$ and has rotational losses of $300$ $W$. Neglecting the rotational losses, find the 
    no-load speed, line current, and power 
    factor for the motor.
\end{frame}
\begin{frame}{Problem 16.15}
    \begin{figure}[h]
        \centering
        \includegraphics[width=0.9\textwidth]{circuit.png}
        \caption{Circuit for problem 16.15}
        \label{nem_ngang}
        \end{figure}
    
\end{frame}
\begin{frame}{Problem 16.15}
  \textbf{Solution:}  Synchronous speed (no-load speed):
    $$n_{s}=\frac{120.60}{4}=1800$$
    The slip:
    $$s=\frac{n_{s}-n_{m}}{n_{s}}=\frac{1800-1740}{1800}=\frac{1}{30}$$
    $$Z_{s}=1+2j+\frac{(0,8j+15)40j}{0,8j+15+40j}=13,7+7,45j=15,6\angle 28,55\degree$$
    Power factor $=\cos{28,55\degree}=87,84\%$ lagging
    $$I_{s}=\frac{V_{s}}{Z_{s}}=\frac{230\angle 0\degree}{15,6\angle 28,55\degree}=14,74\angle -28,55\degree$$
    Line current:
    $$I_{line}=\sqrt{3}I_{s}=14,74\sqrt{3}=25,53 (A rms)$$
    
\end{frame}
\begin{frame}{Problem 16.21}
    A $3$ $hp$ six-pole $60$ $Hz$ delta-connected 
    three-phase induction motor is rated for 
    $1140$ $rpm$, $220$ $V$ $rms$, and $8,58$ $A$ $rms$ (line 
    current) at an $80$ percent lagging power factor. Find the full-load efficiency.
    \\
    \textbf{Solution:}
    \\$$P_{out}=2.746=1492$$
    We have:
    $$V_{s}=V_{line}$$
    $$I_{s}=\frac{I_{line}}{\sqrt{3}}$$
\end{frame}
\begin{frame}{Problem 16.21}
    \begin{align*}
        P_{in}&=3I_{s}V_{s}\cos{\theta}=\frac{3I_{line}}{\sqrt{3}}V_{line}\cos{\theta}=\sqrt{3}I_{line}V_{line}\cos{\theta}\\&=\sqrt{3}.5,72.220.0,8=1743,69 (W)
        \end{align*}
        The efficiency is:
        $$\eta=\frac{1492}{1743,69}.100\%=85,57\%$$
\end{frame}
\begin{frame}{Problem 16.31}
    A certain four-pole $440$ $V$ $rms$ $60$ $Hz$ three
phase delta-connected induction motor has
 $R_{s} = 0,12\Omega, X_{s} = 0,30\Omega,X_{m} = 7.5\Omega,R_{r}^{'} = 0,10\Omega,
 X_{r}^{'}= 0,20 \Omega$.
 Under load, the machine operates with a 
slip of $4$ percent and has rotational losses of 
$2$ $kW$. Determine the power factor, output 
power, copper losses, output torque, and 
efficiency.
\\ \textbf{Solution:}
$$P_{out}=2.746=1492W$$
Synchronous speed:
$$n_{s}=\frac{120f}{p}=\frac{120.60}{8}=900(rpm)$$
\end{frame}
\begin{frame}{Problem 16.31}
    Slip:
    $$s=\frac{n_{s}-n_{m}}{n_{s}}=\frac{900-850}{900}=0,0556$$
    Frequency of motor currents:
    $$f_{r}=s.f_{s}=0,0556.60=3,33 (Hz)$$
    Developed power:
    $$P_{dev}=3\frac{1-s}{s}.R_{r}^{'}I_{r}^{2}=\frac{1-s}{s}P_{r}$$
    Also: $P_{dev}=P_{out}+P_{rot}=1492+100=1592 (W)$
    $$\Rightarrow P_{r}=\frac{s}{1-s}P_{dev}=\frac{0,0556}{1-0,0556}.1592=93,726(W)$$
\end{frame}
\begin{frame}{Problem 16.42}
    A six-pole $60$ $Hz$ synchronous motor is operating with a developed power of $5$ $hp$ and a torque angle of $5 \degree$. 
    Find the speed and developed torque. Suppose that the load increases such that the developed torque doubles.
    Find the new torque angle. Find the pull-out torque and maximum developed power for this machine.
\end{frame}
\begin{frame}{Problem 16.42}
 \textbf{Solution:}
\\ From the problem description we have:
$$P=6W, f=60 Hz, P_{dev1}=5hp, \delta_{1}=5\degree$$
\\The developed power in watts is:
$$P_{dev1}=5.746=3730$$
The synchronous speed is:
$$n_{s}=\frac{120f}{p}=\frac{120.60}{6}=1200 rpm$$
\end{frame}
\begin{frame}{Problem 16.42}
The developed torque is:
$$T_{dev1}=\frac{P_{dev1}}{\omega s}=\frac{P_{dev1}}{ns}.\frac{60}{2\pi}=\frac{3730.1200}{1200}.\frac{60}{2\pi}=29,682 Nm$$
The developed torque is doubled, so we can write:
$$T_{dev2}=2T_{dev1}=2.29,682=59,635 Nm$$
Recall that developed torque is given by following equation:
$$T_{dev}=K.B_{r}.B_{total}.\sin{\delta}$$
\end{frame}
\begin{frame}{Problem 16.42}
    We can use the previous equation to determine new torque angle $\delta_{2}$. But first, we have to calculate the value of constants $K.B_{r}.B_{total}$.
    For the $T_{dev1}$ and $\delta_{1}$, we have:
    $$T_{dev1}=K.B_{r}.B_{total}.\sin{\delta_{1}}\Rightarrow K.B_{r}.B_{total}=\frac{T_{dev1}}{\sin{\delta_{1}}}=\frac{29,682}{\sin{5\degree}}=340,563$$
    For the $T_{dev2}$ and $\delta_{2}$, we can write:
    $$T_{dev2}=K.B_{r}.B_{total}.\sin{\delta_{2}}\Rightarrow \sin{\delta_{2}}=\frac{59,635}{340,563}=0,175$$
    Finally, the torque angle is:
$$\delta_{2}=\arcsin{0,175}=10,08\degree$$
\end{frame}
\begin{frame}{Problem 16.42}
    The pull out torque occurs for torque angle $\delta=90\degree$
$$T_{pull-out}=K.B_{r}.B_{total}.\sin{90\degree}=340,563$$
The maximum developed power is defined as follows:
$$P_{max}=T_{pull-out}\omega.s=T_{pull-out}.ns\frac{2\pi}{60}=42976,41 W$$
\end{frame}
\end{document}
