\documentclass{article}
\usepackage[utf8]{vietnam}
\usepackage[14pt]{extsizes}
\usepackage{amsmath,amsfonts,amsthm}
\usepackage{geometry}
\usepackage{graphicx}
 \geometry{
 a4paper,
 total={170mm,257mm},
 left=20mm,
 top=20mm,
}
\usepackage{enumitem} 
\title{Đề kiểm tra 3}
\begin{document}
\maketitle
\textit{Lưu ý: Thời gian làm bài chỉ có \textbf{60 phút} và \textbf{không} được sử dụng máy tính Casio hay bất cứ tài liệu nào.}
\begin{enumerate}[start=1,label={\bfseries Câu  \arabic*:},leftmargin=1in]
    \item Hai vòi nước $1$ và $2$ được lắp trên thành một bể nước cạn. Khi ta bật vòi nước $1$ trong $2h$ rồi tắt, sau đó bật tiếp vòi nước $2$ thì sau $5h$ bể sẽ đầy. Khi ta bật
cả hai vòi nước cùng một lúc tính từ thời điểm bể cạn thì $4h$ sau bể sẽ đầy nước. Vậy nếu từ thời điểm bể cạn, ta chỉ bật vòi nước $1$ trong $1h$ đầu tiên rồi tắt thì vòi nước $2$ phải được bật trong bao lâu để bể đầy nước?
    \item Giải các bất phương trình sau:
    \begin{enumerate}
        \item $$(\sqrt{3x^2-7x+4}+1)(|x^2-x|-2)<0$$
        \item $$\frac{2}{x}+\frac{3}{12-x}>1$$
    \end{enumerate}
    \item Trong môn Toán, người ta thường dùng kí hiệu $\Sigma$ (đọc là \textit{sigma}) để kí hiệu cho tổng của một chuỗi số tính từ phần tử thứ $0$ là $a_{0}$ đến phần tử
    thứ $n$ là $a_{n}$ như sau:
    $$\sum_{i=0}^{n}a_{i}=a_{0}+a_{1}+a_{2}+...+a_{n}$$
    Hãy chứng minh rằng:
    $$\sum_{i=0}^{90}\cos^2{i^\circ}=\sum_{i=0}^{90}\sin^2{i^\circ}$$
\end{enumerate}
\end{document}

    