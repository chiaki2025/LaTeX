\chapter{Giới thiệu về Microsoft Copilot}
\section{Quá trình phát triển}
Bing là một công cụ tìm kiếm được phát triển bởi Microsoft và ra mắt vào năm 2009. Bing cung cấp
các tính năng đa dạng như trình duyệt web; tìm kiếm hình ảnh, video, tin tức, bản đồ cũng như tìm kiếm bằng giọng nói, dịch thuật và kiểm tra chính tả.
Bing cũng được tích hợp với các ứng dụng khác của Microsoft như Edge, Cortana hay Xbox \cite{link_1}. Từ khi ra mắt cho đến thời điểm hiện tại, Bing luôn cải tiến và nâng cấp
các tính năng để mang lại trải nghiệm tốt hơn cho người dùng. Trong năm 2023, Microsoft đã hợp tác với Open AI và ra mắt Bing Chat vào tháng 2. Bing Chat là một AI chat bot đã được tích hợp với
Chat GPT 4, phiên bản mới nhất của Chat GPT tính đến thời điểm hiện tại. Vào tháng 11, Microsoft đã đổi tên Bing Chat thành Microsoft Copilot (hay còn gọi là Copilot) để tối ưu hóa trải nghiệm của người dùng và có thể
dễ dàng tiếp cận được với cộng đồng hơn \cite{link_2}.
\section{Tính năng}
Copilot có thể hỗ trợ người dùng trong việc trả lời các câu hỏi, tìm kiếm thông tin,
tóm tắt lại những ý chính của một bài viết, bài báo, tin tức hay so sánh giữa các sản phẩm với nhau.
Đặc biệt hơn cả, Copilot có thể tự sáng tạo ra hình ảnh hay đoạn văn theo yêu cầu của người sử dụng\cite{link_3}. Microsoft đã cập nhật thêm các tính năng mới từ tháng 5 năm nay để nâng cao trải nghiệm của người dùng hơn gồm \cite{link_4}:
\begin{itemize}
    \item Lịch sử đoạn chat: Copilot có khả năng lưu trữ lại lịch sử của đoạn chat và hiển thị chúng ở phía bên phải màn hình. Người dùng giờ đây có thể truy cập vào bất kì đoạn chat nào đã được lưu trữ.
    \item Biểu đồ: Copilot có khả năng tạo ra một biểu đồ trực quan dạng cột để minh họa các số liệu thống kê.
    \item Xuất bản: Người sử dụng giờ đây đã có thể tải về các câu trả lời của Bing dưới dạng file Word, PDF,... để lưu trữ và sử dụng trong công việc.
    \item Đề xuất: Khi người dùng nhập vào một từ khóa, Copilot sẽ đề xuất từ khóa có thể tương ứng với nội dung đang được tìm kiếm.
    \item Bảo mật: Khi người dùng hỏi các câu hỏi liên quan đến thông tin cá nhân hay các chủ đề riêng tư, Copilot sẽ tự động không lưu trữ lại các đoạn chat này.
\end{itemize}
\section{Cách sử dụng Copilot}
Để có thể truy cập vào Copilot\cite{link_5}, người dùng có một số cách như:
\begin{itemize}
    \item Truy cập vào trang web \url{https://www.bing.com/} và ấn nút Chat
    \item Tải Microsoft Edge \url{https://www.microsoft.com/edge/launch/try-edge-now?form=MA13I2} và ấn vào icon của Copilot
\end{itemize}
Sau khi truy cập thành công vào Copilot, người sử dụng cần phải chọn giữa các phong cách hội thoại gồm các lựa chọn sau:
\begin{enumerate}
    \item Creative: Copilot sẽ đưa ra các câu trả lời sáng tạo và cởi mở.
    \item Precise: Copilot sẽ đưa ra các câu trả lời với độ chính xác cao.
    \item Balanced: Copilot sẽ đưa ra các câu trả lời với phong cách kết hợp giữa Creative và Precise.
\end{enumerate}
Tiếp theo, người sử dụng cần gõ câu hỏi vào thanh chat "Ask me anything" để bắt đầu sử dụng Copilot. Thế nhưng, hiện nay người dùng chỉ có thể hỏi tối đa 30 câu trong một 
cuộc hội thoại với Copilot. Sau khi hỏi hết 30 câu, người dùng cần phải tạo một cuộc hội thoại mới.

