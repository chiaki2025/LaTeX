\documentclass{article}
\usepackage[utf8]{vietnam}
\usepackage[14pt]{extsizes}
\usepackage{amsmath,amsfonts,amsthm}
\usepackage{geometry}
\usepackage{mathrsfs}

 \geometry{
 a4paper,
 total={170mm,257mm},
 left=20mm,
 top=20mm,
}
\usepackage{enumitem} 
\title{Tín hiệu và hệ thống}
\begin{document}
\maketitle
\section{Giới thiệu về Tín hiệu và hệ thống}
\subsection{Tín hiệu}
\subsubsection{Năng lượng và công suất của tín hiệu}
\begin{enumerate}
    \item Tín hiệu liên tục
    $$E_{\infty}=\int_{-\infty}^{+\infty}|x(t)|^2dt$$
    $$P=\lim_{T\to\infty}\frac{1}{2T}\int_{-T}^{+T}|x(t)|^2dt$$
    \item Tín hiệu rời rạc
    $$E_{\infty}=\sum_{n=-\infty}^{+\infty}|x[n]|^2$$
\end{enumerate}
\section{Hệ thống LTI}
\section{Biến đổi Fourier}
\subsection{Chuỗi Fourier (FS)}
\subsubsection{Chuỗi Fourier cho tín hiệu liên tục (CTFS)}
Khái niệm: Một tín hiệu liên tục và \textbf{tuần hoàn} bất kì có thể được biểu diễn dưới dạng tổng của các tín hiệu tuần hoàn nhỏ hơn:
$$x(t)=\sum_{k=-\infty}^{+\infty}a_{k}e^{jk\omega t}$$
$$a_{k}=\frac{1}{T}\int_{-\infty}^{+\infty}x(t)e^{-jk\omega t}$$
\section{Biến đổi Laplace}
\section{Biến đổi Z}
$$\mathscr{L}(x[n])$$
\begin{itemize}
    \item $$\%$$
\end{itemize}
\end{document}