\documentclass{article}
\usepackage[utf8]{vietnam}
\usepackage[14pt]{extsizes}
\usepackage{amsmath,amsfonts,amsthm}
\usepackage{geometry}
\usepackage{mathrsfs}

 \geometry{
 a4paper,
 total={170mm,257mm},
 left=20mm,
 top=20mm,
}
\usepackage{enumitem} 
\title{Tín hiệu và hệ thống}
\begin{document}
\maketitle
\section{Phương trình sai phân}
\subsection{Biểu diễn hệ thống}
Một hệ thống là một cấu trúc sẽ có một cấu hình chung như này:
\begin{equation*}
    x[n]\rightarrow h[n] \rightarrow y[n]
\end{equation*}
trong đó thì $h[n]$ được gọi là ĐÁP ỨNG XUNG CỦA HỆ THỐNG.
Một hệ thống rời rạc bất kỳ có thể được biểu diễn bằng phương trình sai phân.
\\ Ví dụ: $y[n] = y[n-1] + x[n]$, tức là có thể biểu diễn mọi hệ thống với đầu ra là một hàm số theo đầu vào.
\subsection{Mô tả hệ thống bằng phương trình sai phân}
\subsubsection{Đáp ứng của hệ thống}
Khi phân tích một hệ thống bất kì thì đều phải đi qua 2 giai đoạn: giai đoạn khởi tạo và giai đoạn ổn định.
\\ Vậy ta phải phân tích hệ thống thành hai phần (hai nghiệm) để biểu diễn hai quá trình, nghiệm đầu tiên gọi là nghiệm \textbf{ĐÁP ỨNG TỰ NHIÊN} (kí hiệu là $y_{N}[n]$) va nghiem con lai goi la nghiem
\textbf{DAP UNG LUC} (ki hieu la $y_{F}[n]$). 
\subsubsection{Phan tich dap ung he thong}
\begin{enumerate}
    \item Nghiem dap ung tu nhien duoc tinh nhu sau: nghiem dap ung tu nhien la nghiem thoa man \textit{DAU VAO BANG 0} (tuc la $x[n]=0$) va dieu kien khoi tao cua he thong ($y[-1] = ...., y[-2] = .....$)
    \item Nghiem dap ung luc duoc tinh nhu sau: nghiem dap ung luc la nghiem thoa man \textit{DIEU KIEN KHOI TAO BANG 0} (tuc la $y[-1]=y[-2]=0$) va dau vao cua he thong.
\end{enumerate}
\section{Bai tap}
\subsection{$y[n]+2y[n-1]-3y[n-2]=x[n-1]$ voi $y[-1]=0, y[-2]=3$}
Tim dap ung cua he thong.
\begin{enumerate}
    \item Dap ung tu nhien $y_{N}[n]$ la dap ung thoa man khong co dau vao, tuc la ve phai bang 0, va thoa man dieu kien khoi tao. Nghiem $y_{N}[n]$ phai thoa man: $y[n]+2y[n-1]-3y[n-2]=0$ va $y[-1]=0, y[-2]=3$.
Nhung phuong trinh ma co dang $y[n]+2y[n-1]-3y[n-2]=0$ co dang nghiem la $y_{N}[n] = K.\lambda^{n}$, thay vao thi ta co:
\begin{equation*}
    K.\lambda^{n}+2K\lambda^{n-1}-3K\lambda^{n-2}=0 \Leftrightarrow 1 + 2\lambda^{-1}-3\lambda^{-2}=0
\end{equation*}
  \begin{equation*}
    \lambda = 1 or \lambda = -3
  \end{equation*}
Vay ta co $y_{N}[n] = K_{1}(1)^{n} + K_{2}(-3)^{n}$, tiep tuc la nhin cai dieu kien $y[-1] = 0, y[-2] = 3$ roi thay vao, tinh $K_{1}=\frac{9}{4}$ voi $K_{2}=\frac{27}{4}$
    \item Khong co dap ung luc.
\end{enumerate}
\subsection{$4y[n]-y[n-2]=x[n]$ voi $x[n]=\left(\frac{1}{2}\right)^{n}u[n]$}
\begin{enumerate}
    \item Khong co dap ung tu nhien
    \item Dap ung luc tinh nhu sau: thoa man dau vao voi dieu kien khoi tao bang 0.
    \begin{equation*}
        4y[n]-y[n-2]=\left(\frac{1}{2}\right)^{n}u[n]
    \end{equation*}
    Cach giai nhu sau: an thang luon dau ra la nghiem cua dau vao, tuc la an thang $y_{F}[n]=C\left(\frac{1}{2}\right)^{n}u[n]$, thay vao thi ta co
    \begin{equation*}
        4C\left(\frac{1}{2}\right)^{n}u[n]-C\left(\frac{1}{2}\right)^{n-2}u[n-2]=\left(\frac{1}{2}\right)^{n}u[n]
    \end{equation*}
    Ta co voi $n\geq2$ thi $u[n]=u[n-2]=1$ (cai nay rat quan trong va khong duoc quen), voi $n\geq2$, ta co:
    \begin{equation*}
        4C\left(\frac{1}{2}\right)^{n}-C\left(\frac{1}{2}\right)^{n-2}=\left(\frac{1}{2}\right)^{n}
    \end{equation*}
\end{enumerate}
\end{document}