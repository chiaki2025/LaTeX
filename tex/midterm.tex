% This LaTeX was auto-generated from MATLAB code.
% To make changes, update the MATLAB code and export to LaTeX again.

\documentclass{article}

\usepackage[utf8]{inputenc}
\usepackage[T1]{fontenc}
\usepackage{lmodern}
\usepackage{graphicx}
\usepackage{color}
\usepackage{hyperref}
\usepackage{amsmath}
\usepackage{amsfonts}
\usepackage{epstopdf}
\usepackage[table]{xcolor}
\usepackage{matlab}

\sloppy
\epstopdfsetup{outdir=./}
\graphicspath{ {./23020874_VuHanTin_images/} }

\begin{document}

\matlabtitle{De thi giua ki thuc hanh}

\begin{matlabcode}
% Bai 1: Bieu dien tin hieu roi rac
% a) Ve do thi mien thoi gian va in ra nang luong
n=-5:5;
xn=2.^(n/2-1).*(unitstep(n+5)-unitstep(n-6));
figure;
stem(n,xn,'filled');
title('Do thi tin hieu x[n] trong mien thoi gian');
xlabel('n');ylabel('x[n]');
grid on;
\end{matlabcode}
\begin{center}
\includegraphics[width=\maxwidth{50.77772202709483em}]{figure_0.png}
\end{center}
\begin{matlabcode}
energy=sum(abs(xn).^2)
\end{matlabcode}
\begin{matlaboutput}
energy = 15.9922
\end{matlaboutput}
\begin{matlabcode}
figure;
subplot(2,2,1);
xlat=2.^(-n/2).*(unitstep(-n+5)-unitstep(-n-6))-1;
stem(n,xlat,'filled');
title('Do thi tin hieu x[-n]');
xlabel('n');ylabel('x[-n]');
grid on;
subplot(2,2,2);
xtre=2.^((n+1)/2).*(unitstep(n+6)-unitstep(n-5))-1;
stem(n,xtre,'filled');
title('Do thi tin hieu x[n+1]');
xlabel('n');ylabel('x[n+1]');
grid on;
subplot(2,2,3);
xco=2.^((2*n-2)/2).*(unitstep((2*n-2)+5)-unitstep((2*n-2)-6))-1;
stem(n,xtre,'filled');
title('Do thi tin hieu x[2n-2]');
xlabel('n');ylabel('x[2n-2]');
grid on;
subplot(2,2,4);
xnhan=xn.*unitstep(n);
stem(n,xnhan,'filled');
title('Do thi tin hieu x[n]u[n]');
xlabel('n');ylabel('x[n]u[n]');
grid on;
\end{matlabcode}
\begin{center}
\includegraphics[width=\maxwidth{50.77772202709483em}]{figure_1.png}
\end{center}


\begin{matlabcode}
% Bai 2: Lay mau va bieu dien tin hieu tren mien thoi gian
f1=10;f2=150;fsam=4000;
t=0:1/fsam:1;
xt=(1+0.5*sin(2*pi*f1*t)).*sin(2*pi*f2*t);
fs1=800;fs2=200;
figure;
subplot(2,1,1);
n1=0:1/fs1:1;
xn1=(1+0.5*sin(2*pi*f1*n1)).*sin(2*pi*f2*n1);
stem(n1,xn1,'.');
title('Tin hieu lay mau voi tan so fs1=800Hz');
xlabel('n');ylabel('x1[n]');
grid on;
subplot(2,1,2);
n2=0:1/fs2:1;
xn2=(1+0.5*sin(2*pi*f1*n2)).*sin(2*pi*f2*n2);
stem(n2,xn2,'.');
title('Tin hieu lay mau voi tan so fs2=200Hz');
xlabel('n');ylabel('x2[n]');
grid on;
\end{matlabcode}
\begin{center}
\includegraphics[width=\maxwidth{50.77772202709483em}]{figure_2.png}
\end{center}
\begin{matlabcode}
figure;
subplot(2,1,1);
Xt=fft(xt);
Nt=length(xt);
Mag_Xt=abs(Xt)/Nt;
Arg_Xt=angle(Xt);
wt=linspace(0,2*pi,Nt);
ft=(wt*fsam)/(2*pi);
plot(ft,Mag_Xt);
title('Pho bien do cua tin hieu x(t)');
xlabel('Tan so Hz'); ylabel('|X(f)|');
grid on;
subplot(2,1,2);
plot(ft,Arg_Xt);
title('Pho pha cua tin hieu x(t)');
xlabel('Tan so Hz'); ylabel('<X(f)');
grid on;
\end{matlabcode}
\begin{center}
\includegraphics[width=\maxwidth{50.77772202709483em}]{figure_3.png}
\end{center}
\begin{matlabcode}
Xn1=fft(xn1);
Nn1=length(xn1);
Mag_Xn1=abs(Xn1)/Nn1;
Arg_Xn1=angle(Xn1);
wn1=linspace(0,2*pi,Nn1);
fn1=(wn1*fs1)/(2*pi);
figure;
subplot(2,1,1);
plot(fn1,Mag_Xn1);
title('Pho bien do cua tin hieu x1[n]');
xlabel('Tan so Hz'); ylabel('|X(f)|');
grid on;
subplot(2,1,2);
plot(fn1,Arg_Xn1);
title('Pho pha cua tin hieu x1[n]');
xlabel('Tan so Hz'); ylabel('<X(f)');
grid on;
\end{matlabcode}
\begin{center}
\includegraphics[width=\maxwidth{50.77772202709483em}]{figure_4.png}
\end{center}
\begin{matlabcode}
figure;
Xn2=fft(xn2);
Nn2=length(xn2);
Mag_Xn2=abs(Xn2)/Nn2;
Arg_Xn2=angle(Xn2);
wn2=linspace(0,2*pi,Nn2);
fn2=(wn2*fs2)/(2*pi);
figure;
subplot(2,1,1);
plot(fn2,Mag_Xn2);
title('Pho bien do cua tin hieu x2[n]');
xlabel('Tan so Hz'); ylabel('|X(f)|');
grid on;
subplot(2,1,2);
plot(fn2,Arg_Xn2);
title('Pho pha cua tin hieu x2[n]');
xlabel('Tan so Hz'); ylabel('<X(f)');
grid on;
\end{matlabcode}
\begin{center}
\includegraphics[width=\maxwidth{50.77772202709483em}]{figure_5.png}
\end{center}
\begin{matlabcode}
% Ket qua hien tuong thu duocL Voi tan so lay mau cang thap thi pho bi
% chong (aliasing), dan den tin hieu khoi phuc duoc khong con giong voi tin
% hieu goc. Dong thoi pha cua cac tin hieu lay mau voi tan so thap cung bi
% meo di dang ke.
\end{matlabcode}


\begin{matlabcode}
% Bai 3: He thong tuyen tinh bat bien:
b=[1 2 -1];
a=[1 4 -3];
n=0:20;
figure;
impz(b,a);
grid on;
\end{matlabcode}
\begin{center}
\includegraphics[width=\maxwidth{50.77772202709483em}]{figure_6.png}
\end{center}
\begin{matlabcode}
zi=filtic(b,a,[0 4],[-1 1]);
ystep=filter(b,a,unitstep(n),zi);
figure;
stem(n,ystep,'filled');
title('Dap ung xung bac thang cua he thong');
xlabel('n');ylabel('hst[n]');
grid on;
\end{matlabcode}
\begin{center}
\includegraphics[width=\maxwidth{50.77772202709483em}]{figure_7.png}
\end{center}
\begin{matlabcode}
zplane(b,a);
grid on;
\end{matlabcode}
\begin{center}
\includegraphics[width=\maxwidth{50.77772202709483em}]{figure_8.png}
\end{center}
\begin{matlabcode}
% He thong nay khong on dinh
xn=exp(-n).*unitstep(n);
yxn=filter(b,a,xn,zi);
figure;
stem(n,yxn,'filled');
title('Dau ra cua tin hieu x[n]');
xlabel('n');ylabel('y[n]');
grid on;
\end{matlabcode}
\begin{center}
\includegraphics[width=\maxwidth{50.77772202709483em}]{figure_9.png}
\end{center}
\begin{matlabcode}
% He thong nhan qua khong on dinh, de he thong on dinh, ta phai dieu chinh lai
% vector a
a1=[1 1/3 0];
zplane(b,a1);
grid on;
\end{matlabcode}
\begin{center}
\includegraphics[width=\maxwidth{50.77772202709483em}]{figure_10.png}
\end{center}


\begin{matlabcode}
% Bai 4: Loc IIR
fsam=1000;f1=50;f2=100;f3=200;
n=0:1/fsam:1;
xn=sin(2*pi*f1*n)+sin(2*pi*f2*n)+sin(2*pi*f3*n);
yn=xn+randn(size(xn));
figure;
Xn=fft(xn);
Nx=length(xn);
MagXn=abs(Xn)/Nx;
subplot(2,1,1);
wx=linspace(0,2*pi,Nx);
fx=(wx*fsam)/(2*pi);
plot(fx,MagXn);
title('Pho bien do tin hieu x[n]');
xlabel('Tan so Hz'); ylabel('|X(f)|');
grid on;
subplot(2,1,2);
Yn=fft(yn);
Ny=length(yn);
MagYn=abs(Yn)/Ny;
wy=linspace(0,2*pi,Ny);
fy=(wy*fsam)/(2*pi);
plot(fy,MagYn);
title('Pho bien do tin hieu y[n]');
xlabel('Tan so Hz'); ylabel('|Y(f)|');
grid on;
\end{matlabcode}
\begin{center}
\includegraphics[width=\maxwidth{50.77772202709483em}]{figure_11.png}
\end{center}


\matlabheading{Ta can thiet ke bo loc BP Chebyshev voi cac thong so sau: fpass=[80 170], fstop= [50 200], Rpass=1, Rstop=30}

\begin{matlabcode}
fpass=[80 170]; fstop=[50 200]; Rpass=1; Rstop=30;
wpass=2*fpass/fsam; wstop=2*fstop/fsam;
[n_bac Wn]=cheb1ord(wpass,wstop,Rpass,Rstop);
[b a]=cheby1(n_bac,Rpass,Wn);
yloc=filter(b,a,yn);
figure;
subplot(2,1,1);
stem(n,yloc,'.');
title('Tin hieu yloc[n] sau khi loc o mien thoi gian');
xlabel('n'); ylabel('yloc[n]');
grid on;
subplot(2,1,2);
Yn=fft(yloc);
Nloc=length(yloc);
wy=linspace(0,2*pi,Nloc);
fy=fsam*wy/(2*pi);
MagY=abs(Yn)/Nloc;
plot(fy,MagY);
title("Pho cua tin hieu yloc[n]");
xlabel('Tan so Hz'); ylabel('|Yloc(f)|');
grid on;
\end{matlabcode}
\begin{center}
\includegraphics[width=\maxwidth{50.77772202709483em}]{figure_12.png}
\end{center}
\begin{matlabcode}

function u=unitstep(n)
u=(n>=0);
end

\end{matlabcode}

\end{document}