\documentclass[8pt]{beamer}
\usepackage{tikz}
\usepackage[utf8]{vietnam}
\usepackage{amsmath}
\usepackage{graphicx}
\usepackage{mathrsfs}
\usepackage{amssymb,amsfonts,amsthm}
\usepackage{wrapfig}
\usepackage{hyperref}
\usetheme{Copenhagen}
\usecolortheme{spruce}
\setbeamertemplate{navigation symbols}{}
\setbeamertemplate{headline}{}
\setbeamertemplate{footline}{}
\title[Kết quả nghiên cứu tuần 3]
{Kết quả nghiên cứu tuần $3$}
\subtitle{Phòng thí nghiệm Thông tin Vô tuyến}
\author[Phòng thí nghiệm thông tin Vô tuyến]
{Tín Vũ}
\date[VLC 2021] % (optional)
{tinvu1309@gmail.com}
\begin{document}
\frame{\titlepage}
\begin{frame}{Table of Contents}
\tableofcontents
\end{frame}
\begin{frame}{Reference documents}
\section{Reference documents}
The document used for study in this lab report is: Antenna Theory (chapter 2 + 3 + 4).
\end{frame}
\begin{frame}{Magnetic vector potential}
\section{Magnetic vector potential}
In the previous lab slide, we only derive some very basic antenna parameters are $P$, $U$, $D$, $G$ based on 
the given $W(\phi,\theta)$ function. So, how to determine $W(\phi,\theta)$ function?
\\ From the Maxwell's equation in differential form:
\begin{equation*}
\begin{split}
\nabla\cdot\vec{D}&=\rho\\
\nabla\cdot\vec{H}&=0\\
\nabla\times\vec{E}&=-\frac{\partial \vec{B}}{\partial t}=-j\omega \vec{B}=-\frac{\partial}{\partial t} \mu_{0}\vec{H}\\
\nabla\times\vec{H}&=\vec{J}+\frac{\partial \vec{D}}{\partial t}=\vec{J}+j\omega \varepsilon_{0}\vec{E}=\vec{J}+j\omega\vec{D}
\end{split}
\end{equation*}
We introduce the new parameter: $\vec{D}=\varepsilon_{0}\vec{E}$ (electric displacement).
\\ So how to design current density $\vec{J}$ to satisfies our required antenna characteristic? Or equivalently, given $\vec{J}$,
determine $\vec{E}$ and $\vec{H}$; and finally use the complex power density of Poynting vector (irradiance) equation to yield $W(\phi,\theta)$.
\\ Set an auxiliary vector $\vec{A}$ (also can be named as magnetic vector potential) defined as:
$$\vec{B}=\nabla \times A$$
From the third equation:
$$\nabla\times\vec{E}=-j\omega(\nabla\times\vec{A})\Rightarrow \nabla\times(\vec{E}+j\omega\vec{A})=0$$
The right term is equal to the gradient descent of $\phi$: $$\vec{E}+j\omega\vec{A}=-\nabla \phi$$
\end{frame}
\begin{frame}{Magnetic vector potential}
From the fourth equation:
\begin{equation*}
\begin{split}
\nabla\times\vec{H}&=\vec{J}+j\omega \varepsilon_{0}\vec{E}=\vec{J}+j\omega \varepsilon_{0}(-j\omega\vec{A}-\nabla\phi)\\
\Rightarrow \frac{1}{\mu_{0}}(\nabla\times\nabla\times\vec{A})&=\vec{J}+j\omega \varepsilon_{0}(-j\omega\vec{A}-\nabla\phi)\\
\Rightarrow \frac{1}{\mu_{0}}[\nabla(\nabla\cdot \vec{A})-\nabla^2\vec{A}]&=\vec{J}+\omega^2\varepsilon_{0}\vec{A}-j\omega\varepsilon_{0}\nabla\phi\\
\Rightarrow \nabla(\nabla\cdot \vec{A})-\nabla^2\vec{A}&=\mu_{0}(\vec{J}+\omega^2\varepsilon_{0}\vec{A}-j\omega\varepsilon_{0}\nabla\phi)\\
\end{split}
\end{equation*}
After some reduction steps I don't understand, we yield:
$$\nabla^2\vec{A}+k_{0}^2\vec{A}=-\mu_{0}\vec{J} \quad\text{($k_{0}$ is a constant)}$$
This equation has one solution:
$$\vec{A}(\vec{r})=\frac{\mu_{0}}{4\pi}\int \vec{J}(\vec{r'})\frac{e^{-jk|\vec{r}-\vec{r'}|}}{|\vec{r}-\vec{r'}|}dv'$$
Somehow we can neglect gradient of $\nabla\phi$, so $\vec{E}$ and $\vec{H}$ can easily be determined in term of $\vec{A}$:
\begin{equation*}
    \begin{cases}
    \vec{E}=-j\omega\vec{A}\\
    \vec{H}=\frac{1}{\mu_{0}}\nabla\times\vec{A}=\alert{\frac{\hat r\times\vec{E}}{\eta }=\frac{\hat r\times(\vec{E_{\theta}}\hat \theta+\vec{E_{\phi}}\hat\phi)}{\eta}}\\
    \end{cases}
\end{equation*}
\end{frame}
\begin{frame}{Magnetic vector potential}
In short, we summarize 4 steps to find $\vec{E}$ and $\vec{H}$ in far-field zone with given current density $\vec{J}$:
\begin{enumerate}
    \item[1] Find magnetic vector potential $\vec{A}(\vec{r'})$ from $\vec{J}$ and distance $\vec{r'}$.
    \item[2] Seperate $\vec{A_{theta}}$ and $\vec{A_{\phi}}$ components.
    \item[3] Substitute results into each $\vec{E}$ components:
    \begin{equation*}
        \begin{split}
            \vec{E}_{r}&=0\\
            \vec{E_{\theta}}&=-j\omega\vec{A_{\theta}}\\
            \vec{E_{\phi}}&=-j\omega\vec{A_{\phi}}
        \end{split}
    \end{equation*}
    \item[4] Find the magnetic intensity function: $$\vec{H}=\frac{\vec{r}\times\vec{E}}{\eta}$$
\end{enumerate}
From $\vec{E}$ and $\vec{H}$ above, we can easily deduce: $$\vec{W}=\frac{1}{2}\vec{E}\times\vec{H^*}$$
Now we apply our steps to analyze the dipole, which is the simplest antenna.
\end{frame}
\end{document}