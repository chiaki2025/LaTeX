\documentclass{article}
\usepackage[utf8]{inputenc}
\usepackage{amsmath,amsthm,amsfonts}
\usepackage{graphicx}

\newtheorem{theorem}{Theorem}[section]
% \nemtheorem{corollary}{Corollary}[theorem]

\newcommand{\R}{\mathbb{R}}
\newcommand{\cv}[2]{\begin{bmatrix}
    #1\\
    #2\\
\end{bmatrix}}

\title{Hello World}
\author{Chiaki}
\date{Nov 19}
\begin{document}
\maketitle
\section*{Introduction}
\begin{enumerate}
\item Let's begin with a \textbf{formula}: $e^{i\pi}+1=0$
\item \textit{But we can also do}
$$\lim_{n\to+\infty}\left(1+\frac{1}{n}\right)^n=\lim_{n\to+\infty}\frac{n}{\sqrt[n]{n!}}=\lim_{x\to 0}\left(1+\frac{1}{x}\right)^{x}=e$$
\item We can do another:
\begin{align}
e&=\sum_{n=0}^{+\infty}\frac{1}{n!}\\\text{Like and subscribe}&=\lim_{n\to+\infty}\left(1+\frac{1}{n}\right)^{n}
\end{align}
\end{enumerate}
\section*{\underline{More formula}}
$$\int_{a}^{+\infty}f(x)dx$$
$$\iint_{a}^{b}$$
$$\vec{v}=<v_1,v_2>$$
$$\begin{bmatrix}
1 & 2 &3\\
4 & 5 & x \\
\end{bmatrix}$$
$$\begin{bmatrix}
    1&2&3&4\\
    5&x&y&z\\
\end{bmatrix}$$
$$\int_{-\infty}^{+\infty}\sin{x}dx$$
Define $\pi$ value: $$\pi=3.1415$$
Otherwise, we can define:
\begin{equation}
    e=2.71
\end{equation}
\begin{equation}
\label{pi}
    \pi=3.14 
\end{equation}
Matrix \ref{pi} is cool!!!!
\begin{figure}
    \centering
    \includegraphics[width=1\textwidth]{sakuya.png}
    \caption{A cute cat.}
    \label{fig:cat}
    \end{figure}
\begin{figure}[h]
    \centering
    \includegraphics[width=1\textwidth]{pantsu.png}
    \caption{A cute pantsu}
\end{figure}
\begin{equation}
\begin{split}
S_{n}&=\lim_{n\to+\infty}\sum_{n=1}^{+\infty}\frac{1}{n}\\&=+\infty
\end{split}
\end{equation}
\newpage
\section{More Trick}
\begin{table}
\caption{A nifty table}
\begin{center}
\begin{tabular}{|c|c|}
    \hline
    1&2 \\ \hline
    3a&4b \\ \hline
\end{tabular}
\end{center}
\end{table}
\includegraphics[width=0.5\textwidth]{sakuya.png}
\begin{theorem}[Youtube]
We should like and subscribe 
\begin{proof}
    Check out Visual Studio Code please
\end{proof}
\end{theorem}
\begin{theorem}
    You should ring the notification too
\begin{proof}
    I don't think so
\end{proof}
\end{theorem}
Real numbers symbol: $\mathbb{R}$ \\
Real number: $\R$ \\
I can create a matrix: $\cv{1}{2}$
\end{document}